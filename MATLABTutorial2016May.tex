%MATLAB Tutorial for Summer 2016 Research 
\documentclass[12pt, compress,xcolor={usenames,dvipsnames}]{beamer} %slides and notes
\usepackage{amsmath,datetime,xmpmulti,mathtools,bbm,array,booktabs,alltt,xspace,mathabx,pifont,tikz,graphicx}
\usepackage[autolinebreaks]{mcodefred}
\usetikzlibrary{arrows}
\usetheme{FJHSlim}

%\setlength{\parskip}{2ex}
\setlength{\arraycolsep}{0.5ex}

\input FJHDef.tex
\newcommand{\Matlab}{MATLAB\xspace}




\begin{document}
	\tikzstyle{every picture}+=[remember picture]
	\everymath{\displaystyle}
	
	\title[Illinois Institute of Technology]{\Matlab (MATrix LABoratory) Tutorial}
	\author[Fred J. Hickernell]{Fred J. Hickernell}
	\date[May 23, 2016]{May 23, 2016}
	\frame[label=title]{\titlepage}
	
	\section{Features}
	\begin{frame}\frametitle{\Matlab's Features}
		
		\centerline{\includegraphics[width = 11.7cm]{ProgramsImages/MATLABScreenShot.png}}
		\begin{itemize}
						
			\item All variables are arrays (vectors, matrices, \ldots) by default
			
			\item Operators and functions take matrix inputs
			
			\item Variables need not be declared
			
			\item Help and documentation available
			
			\item Command window
			
			\item Editor for writing scripts (programs) and functions
			
			\item Object-oriented (you can define your own classes)
			
			\item Scripts can be published as nicely looking pdf or html files
			
		\end{itemize}
		
	\end{frame}

\section{Commands}
	
\begin{frame}[fragile]\frametitle{Basic Commands in the Command Window}
\begin{alltt}
>> x = 3:9 %set x to be a vector of inputs
x =
     3     4     5     6     7     8     9
>> x(4:6) %display the 4th through 6th elements
ans =
    6     7     8
>> y = [-2 3 4 6 13 14 16] %a vector of outputs
y =
    -2     3     4     6    13    14    16
>> plot(x,y,'-','linewidth',3) %connected by line segments
>> plot(x,y,'.k','markersize',30) %plot points as dots
\end{alltt}
\end{frame}

\section{Scripts}

\begin{frame}[fragile]\frametitle{Commands in a Script for Safekeeping and Reuse}
	\vspace{-8ex}
\lstinputlisting[linerange = {1-15}]{ProgramsImages/FittingData.m}
\end{frame}

\begin{frame}[fragile]\frametitle{Commands in a Script for Safekeeping and Reuse}
	\vspace{-8ex}
	\lstinputlisting[linerange = {17-25}]{ProgramsImages/FittingData.m}
\end{frame}

\section{Functions}
	
\begin{frame}[fragile]\frametitle{A Function Can Be Called in Other Programs}
	\vspace{-8ex}
	\lstinputlisting[linerange = {1-10}]{ProgramsImages/dataFit.m}	
\end{frame}
	
\begin{frame}[fragile]\frametitle{A Function Can Be Called in Other Programs}
	\vspace{-8ex}
	\lstinputlisting[firstline = 12]{ProgramsImages/dataFit.m}	
\end{frame}


\section{Classes}

\begin{frame}[fragile]\frametitle{Every \Matlab Variable Belongs to a Class}
	\begin{alltt}
		>> A = [1 2; 3 4]
		A =
		1     2
		3     4
		>> fred = 'elaine'
		fred =
		elaine
		>> Hartur = \{'Brazil','Olympics'\}
		Hartur = 
		'Brazil'    'Olympics'
			\end{alltt}
\end{frame}

\begin{frame}[fragile]\frametitle{Every \Matlab Variable Belongs to a Class}
	\vspace{-5ex}
	\begin{alltt}
		>> whos
		Name        Size            Bytes  Class     Attributes
		
		A           2x2                32  double              
		Hartur      1x2               252  cell                
		fred        1x6                12  char                
	\end{alltt}
	In \Matlab you may define your \alert{own} classes.
\end{frame}


\begin{frame}[fragile]\frametitle{A Class Has Properties}
	\vspace{-8ex}
	\lstinputlisting[linerange = {1-12}]{ProgramsImages/dataFitClass.m}	
\end{frame}

\begin{frame}[fragile]\frametitle{A Class Has Methods}
	\vspace{-8ex}
	\lstinputlisting[linerange = {14-26}]{ProgramsImages/dataFitClass.m}	
\end{frame}

\begin{frame}[fragile]\frametitle{A Class Has Methods}
	\vspace{-8ex}
	\lstinputlisting[linerange = {28-36}]{ProgramsImages/dataFitClass.m}	
\end{frame}

\begin{frame}[fragile]\frametitle{A Class Has Methods}
	\vspace{-8ex}
	\lstinputlisting[linerange = {38-49}]{ProgramsImages/dataFitClass.m}	
\end{frame}

\section{bsxfun}

\begin{frame}{\texttt{bsxfun}}
	\vspace{-4ex}
	Supose that you have a row vectors, \mcode{x}, and you want to make the matrix \mcode{A = (a_ij) = (x_i^j)}.  The function \mcode{bsxfun} does this quickly:
	\vspace{-2ex}
	\lstinputlisting[linerange = {27-34}]{ProgramsImages/FittingData.m}
\end{frame}

\section{Publishing}

\begin{frame}{Publishing \Matlab Scripts}
		The \mcode{publish} command in \Matlab generates html (or pdf) files from \Matlab scripts that have a beautiful appearance.  Try 
		
		\bigskip
		
		\mcode{publish FittingData}
		
		\bigskip
		
		Then look at the file \texttt{FittingData.html} in the \texttt{html} folder.
		
\end{frame}

\section{Resources}

\begin{frame}{Resources}
	Check out the \Matlab Tutorials at \beamerbutton{\href{http://www.mathworks.com/academia/student_center/tutorials/mltutorial_launchpad.html}{Mathworks \Matlab Tutorials}}
	
	
	\bigskip
	
	Come to the \alert{Advanced \Matlab Tutorial} here at 9 AM, tomorrow, Tuesday, May 24.
	
\end{frame}


\end{document}



