%Git Tutorial for Summer 2016 Research 
\documentclass[12pt, compress,xcolor={usenames,dvipsnames}]{beamer} %slides and notes
\usepackage{amsmath,datetime,xmpmulti,mathtools,bbm,array,booktabs,alltt,xspace,mathabx,pifont,tikz,graphicx}
\usepackage[autolinebreaks]{mcodefred}
\usetikzlibrary{arrows}
\usetheme{FJHSlim}

%\setlength{\parskip}{2ex}
\setlength{\arraycolsep}{0.5ex}

\input FJHDef.tex
\newcommand{\Matlab}{MATLAB\xspace}




\begin{document}
	\tikzstyle{every picture}+=[remember picture]
	\everymath{\displaystyle}
	
	\title[Illinois Institute of Technology]{Git (Global Information Tracker) Tutorial}
	\author[Fred J. Hickernell]{Fred J. Hickernell}
	\date[May 23, 2016]{May 23, 2016}
	\frame[label=title]{\titlepage}
	
	
\section{Features}
	\begin{frame}\frametitle{Git's Features}
		
		\vspace{-3ex}
		
		\begin{itemize}
						
			\item Git is an open source protocol for \alert{source (version) control management} of text files, such as programs and articles.
			
			\item It facilitates
			
			\begin{itemize}
				\item A \alert{record of changes} so that one may go back to earlier versions,
				
				\item \alert{Collaborative} authorship, where multiple people can be editing at the same time and those changes are successfully merged, and
				
				\item A method for \alert{resolving conflicts} if inconsistent changes are made simultaneously.
			\end{itemize}
			
			\item Copies of the repository (all the files in the same big group) are \alert{distributed} in multiple locations that update each other.  There is typically one remote repository that contains the nearly latest copy.
			 
			
		\end{itemize}
		
	\end{frame}
	
	\begin{frame}\frametitle{Git's Features}

		\vspace{-3ex}
		
		\begin{itemize}
						
			\item Documents may be
			
			\begin{description}
				\item[Untracked] Existing only on your local copy, but not in other copies of the repository,
				
				\item [Edited] Being changed, but not yet ready to go further,
				
				\item[Staged] Ready to become official,
				
				\item[Committed] An official part of your local copy of the repository, but not yet a part of the remote repository.
				
				\item[Pushed] Pushed to the remote copy(ies) of the repository, say on GitHub or Bitbucket.  
			\end{description}
			
			\item The repository may split into \alert{branches} that can also be merged again so that one group can work on something without disturbing others.
			
			
		\end{itemize}
		
	\end{frame}

\section{Set-Up}

\begin{frame}\frametitle{To Get Started \ldots}
	
	\vspace{-4ex}

\begin{itemize}
	
	\item Create an account for yourself at the GitHub remote repository \href{http://github.com}{\nolinkurl{github.com}}.
	
	\item Download the Git client SourceTree client application at \href{http://sourcetreeapp.com}{\nolinkurl{sourcetreeapp.com}}.
	
\end{itemize}

\end{frame}

\begin{frame}\frametitle{Setting Up SourceTree}
	
	\vspace{-4ex}
	
	\begin{itemize}
				
		\item Open the Preferences pane in SourceTree and add your name, email address, and preferred folder for saving copies of repositories.
		
		\includegraphics[width =11cm]{ProgramsImages/SourceTreePreferences.png}
	\end{itemize}
	
\end{frame}

\begin{frame}\frametitle{Setting Up SourceTree}
	
	\vspace{-4ex}
	
	\begin{itemize}
		
		\item Chose \texttt{Window} $\rightarrow$ \texttt{Show Repository Browser} and then click the little wheel in the top right corner.  Add your GitHub account credentials.
		
		\includegraphics[width =11cm]{ProgramsImages/SourcTreeRepositoryBrowserSettings.png}
	\end{itemize}
	
\end{frame}

\section{Cloning}

\begin{frame}\frametitle{Cloning an Existing Repository}
	
	To get a copy of an existing repository (on some remote)
	
	\begin{itemize}
		
		\item Chose \texttt{File} $\rightarrow$ \texttt{New/Clone} $\rightarrow$ \texttt{Clone from URL}.
		
		\item Type in \href{http://github.com/fjhickernell/GitHubTutorial.git}{\nolinkurl{http://github.com/fjhickernell/GitHubTutorial.git}}, which is a repository that I have set up for this tutorial.
		
	\end{itemize}
	
	Now you should see in the directory that you created on your own machine the file \texttt{SampleMFile.m}.
	
	Open this file and run it in \Matlab.
	
\end{frame}

\section{Changes}

\begin{frame}\frametitle{Editing Files in a  Repository}
	
	\emph{For this next part I will do it myself because you are not collaborators and cannot push changes to my repository.}
	
	\begin{itemize}
		
		\item After making some changes to a file, I see that is \alert{Uncommitted Changes} appear.
		
		\item To make these changes a part of the repository record, I need to \alert{Stage}, \alert{Commit}, and \alert{Push}.
		
	\end{itemize}
	
	
\end{frame}

\section{New Repository}

\begin{frame}\frametitle{Creating a New Repository}
		
	\begin{itemize}
		
		\item To create your own repository in your own account choose  \texttt{File} $\rightarrow$ \texttt{New/Clone} $\rightarrow$ \texttt{Create Remote Repository}.
		
		\item Enter the name of the repository that you want to create.
		
		\item Find that repository on the Repository Browser.
		
		\item Clone a copy to your local machine.
		
		\item Create a new text file or \Matlab program in your local machine.
		
		\item \alert{Stage}, \alert{Commit}, and \alert{Push} it to your remote repository.
		
	\end{itemize}
	
\end{frame}

\section{Collaborators}

\begin{frame}\frametitle{Adding Collaborators}
			
	
	No one can write to your remote repository unless they are added as a collaborator.  
	
	\begin{itemize}
		
		\item Go to your GitHub account and choose your new repository.
		
		\item  Choose \texttt{Settings}
		
		\includegraphics[width = 11cm]{ProgramsImages/GitHubRemoteRepository.png}
				
		
	\end{itemize}
	
\end{frame}

\begin{frame}\frametitle{Adding Collaborators}
	\vspace{-4ex}
	
	\begin{itemize}
		
		
		\item The choose \texttt{Collaborators} and add a partner to your repository.
		
		\includegraphics[width = 11cm]{ProgramsImages/GitHubRemoteCollaborators}
		
		\item Now have your partner clone your repository, edit your file, and push it back to the repository.
		
		\item (You should now have at least three repositories:  mine, yours, and your partners.)
		
		
	\end{itemize}
	
\end{frame}

\section{Conflicts}

\begin{frame}\frametitle{Resolving Conflicts}
	\vspace{-4ex}
	
	\begin{itemize}
		
		
		\item Now, arrange for you and your partner to change the same line of one of your files, and both of you push the changes.  You should get a \alert{conflict}.
		
		\item The one that got the conflict warning should not choose \texttt{Actions} $\rightarrow$ \texttt{Resolve Conflicts} $\rightarrow$ \texttt{Launch External Merge Tool}.
		
		\item After resolving the conflict by choosing the left or right \alert{save} your merge, \alert{quit} the File Merge application, and \alert{Commit} the result.
				
		
	\end{itemize}
	
\end{frame}

\section{More}

\begin{frame}\frametitle{Reminders}
	
	\begin{itemize}
		
		
		\item Git is better than Dropbox or Google Drive because it keeps track of changes and smoothly handles changes by multiple people at the same time.
		
		
		\item Because Git only stores changes, it is best for text files, not binary files.  
		
		\item Git-LFS \href{https://git-lfs.github.com}{\nolinkurl{git-lfs.github.com}}  is a way to track binary and other large files.  It takes a bit to set up, but is supported by the SourceTree application.  
		
		
		
	\end{itemize}
	
\end{frame}

\begin{frame}\frametitle{Further Topics}
	If there is time we may talk about
	
	\begin{itemize}
		
		
		\item Going back to an earlier place
		
		\item Feature Branches
		
		
		
	\end{itemize}
	
\end{frame}


\end{document}



